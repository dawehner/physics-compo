\documentclass[a4paper,10pt]{article}
\usepackage[utf8x]{inputenc}

\usepackage{array}
\usepackage{graphicx}
\usepackage[sumlimits,intlimits,namelimits]{amsmath}
\usepackage{amssymb,amscd}
\usepackage{dsfont}
\usepackage{mathpazo}
\usepackage[scaled = 0.95]{helvet}
\usepackage{courier}
\usepackage{amsmath}

\newcommand{\bra}[1]{\left\langle #1 \right|}
\newcommand{\ket}[1]{\left| #1\right\rangle}
\newcommand{\skal}[2]{\left\langle #1 | #2 \right\rangle}
\newcommand{\mskal}[2]{\left\langle #1 , #2 \right\rangle}
\newcommand{\dx}{\:\text{\normalfont{d}}}
\newcommand{\e}[1]{\text{\normalfont{e}}^{#1}}
\newcommand{\eps}{\varepsilon}
\newcommand{\N}{\mathds N}
\newcommand{\Z}{\mathds Z}
\newcommand{\R}{\mathds R}
\newcommand{\Q}{\mathds Q}
\newcommand{\C}{\mathds C}
\newcommand{\K}{\mathds K}
\newcommand{\V}{\mathds V}
\newcommand{\I}{\Im}
\newcommand{\Loes}[1]{\mathds L &:=\left\{ #1 \right\}}
\newcommand{\prt}[2]{\frac{\partial #1}{\partial #2}}
\newcommand{\diff}[2]{\frac{\text{d} #1}{\text{d} #2}}
\newcommand{\Diff}[1]{\frac{\text{d}}{\text{d} #1}}
\newcommand{\svec}[1]{\begin{pmatrix} #1 \end{pmatrix}}
\newcommand{\detm}[1]{\begin{vmatrix} #1 \end{vmatrix}}
\newcommand{\BegLGS}[1]{\left\{\begin{array}{#1}}
\newcommand{\EndLGS}{\end{array}\right.}
\newcommand{\abs}[1]{\left|#1\right|}
\newcommand{\norm}[1]{\left\|#1\right\|}
%----------Textkommandos-----------------
\newcommand{\DEF}[1]{\subsection[Definition: #1]{Definition (#1):}}
\newcommand{\KOR}[1]{\subsection[Korollar: #1]{Korollar (#1):}}
\newcommand{\LEMMA}[1]{\subsection[Lemma: #1]{Lemma (#1):}}
\newcommand{\BEM}{\paragraph{Bemerkung:}}
\newcommand{\BSP}{\paragraph{\textcolor{HeadColor}{Beispiel.}}}
\newcommand{\SATZ}[1]{\subsection[Satz: #1]{\textsc{Satz (#1):}}}
%----------Alphabetische Sortierung------
\newcounter{ale}
\newcommand{\abc}{\item[\alph{ale})]\stepcounter{ale}}
\newenvironment{liste}{\begin{itemize}}{\end{itemize}}
\newcommand{\aliste}{\begin{liste} \setcounter{ale}{1}}
\newcommand{\zliste}{\end{liste}}
\newenvironment{abcliste}{\aliste}{\zliste}

%opening

\begin{document}

\section{Aufgabe 7}

Zu zeigen ist, dass die folgende Variante der partitionierten Euler-Methode symplektisch ist.

\begin{align*}
 q^{n+1} &= q^n + \Delta t \prt H p \left(q^{n + 1}, p^n \right) = q^n + h \prt H p \left(q^{n + 1}, p^n \right) =  \\
 p^{n+1} &= p^n - \Delta t \prt H q \left(q^{n + 1}, p^n \right) = p^n - h \prt H q \left(q^{n + 1}, p^n \right)\\
\end{align*}

Wie im Skript unter 3.53 gezeigt wurde, reicht es wenn Poisson-Klammern 1 ergeben.
Dazu müssen die partiellen Ableitungen berechnet werden:

\begin{align*}
q^{n+1}_q &= 1 + h H_{,pq} \cdot q^{n+1}_{,q} + h H_{,pp} p^n_{,q} =  1 + h H_{,pq} \cdot q^{n+1}_{,q} \\
q^{n+1}_p &= 0 + h H_{,pq} \cdot q^{n+1}_{,p} + h H_{,pp} p^n_{,p} = h H_{,pq} \cdot q^{n+1}_{,p} + h H_{,pp} \\
p^{n+1}_q &= 0 - h H_{,qq} \cdot q^{n+1}_{,q} - h H_{,qp}  \cdot p^{n}_{,q} = - h H_{,qq}  \cdot q^{n+1}_{,q} \\
p^{n+1}_p &= 1 - h H_{,qq} \cdot q^{n+1}_{,p} - h H_{,qp} \\
\end{align*}
Die ersten beiden Gleichungen lassen sich direkt auf eine explizite Form bringen(d.h. nicht direkt von q und p abhängig sondern nur von Ableitungen von H).
\begin{align*}
q^{n+1}_q &= \frac{1}{1 - h H_{,pq}} \\
q^{n+1}_p &= \frac{h H_{,pp}}{1 - h H_{,pq}} \\
\Rightarrow
p^{n+1}_q &= - h H_{,qq}  \cdot q^{n+1}_{,q} = - h H_{,qq}  \cdot \frac{1}{1 - h H_{,pq}} \\
p^{n+1}_p &= 1 - h H_{,qq} \cdot q^{n+1}_{,p} - h H_{,qp} = 1 - h H_{,qp} - h H_{,qq} \cdot \frac{h H_{,pp}}{1 - h H_{,pq}}
\end{align*}
Damit ergibt sich für die Poisson-Klammer:
\begin{align*}
\left\{
  q^{n+1}, p^{n+1}
\right\} &= q^{n+1}_{,q} \cdot p^{n+1}_{,p} - q^{n+1}_{,p} \cdot p^{n+1}_{,q} \\
&=
\left(\frac{1}{1 - h H_{,pq}}\right) \cdot
\left(1 - h H_{,qp} - h H_{,qq} \cdot \frac{h H_{,pp}}{1 - h H_{,pq}}\right)
-
\left(\frac{h H_{,pp}}{1 - h H_{,pq}}\right)
\left(\frac{- h H_{,qq}}{1 - h H_{,pq}} \right) \\
&= 1 - \frac{h^2 H_{,qq} H_{,pp}}{(1- h H{,pq})^2}
+ \frac{h^2 H_{,qq} H_{,pp}}{(1- h H{,pq})^2} = 1 \\
\end{align*}

Damit ist die obige Methode symplektisch.

\section{Aufgabe 8}

\subsection{(b)}
Im Vergleich zu den Plots von Blatt2 fällt auf, dass die Energie und der Drehimpuls erhalten bleibt.
Zwar schwankt der Wert innerhalb einer Periode, aber nach einer ganzen Periode ist er wieder konstant.

\subsection{(c)}
Zur Bestimmung der Integrationsgenauigkeit und den Konvergenzeigenschaften
konnte man in Aufgabe 2 noch auf Erhaltungsgrößen aufbauen. Da diese nun erhalten
sind macht es weniger Sinn.\\

Stattdessen kann man mit Blatt1 den genauen Ort der Teilchen berechnen. Diese
sind leicht unterschiedlich zum berechneten Wert mit Leapfrog.

Nun erhöht man die Interval-Anzahl und schaut um wieviel besser der Unterschied wird.
Damit lässt sich die Integrationsgenauigkeit herausfinden.



\end{document}
